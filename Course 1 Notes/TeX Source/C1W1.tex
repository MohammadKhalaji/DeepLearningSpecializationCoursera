\textbf{Sequence Data:} Audio signals and natural language sentences are considered sequence data because they have a "temporal" component. For these kinds of inputs, we often use Recurrent Neural Networks (RNNs). 


\textbf{Hybrid Neural Networks:} Consider a situation where we have an image alongside with radar information and we want to predict the position of other cars. In this example, we need a CNN for processing the image, in addition to a standard NN in order to process the radar info. As a result, we have to make a custom hybrid neural network. 

\textbf{Neural Networks' take off reasons}
\begin{itemize}
	\item \textbf{Data:} more data is available now, and the performance of neural networks is highly dependent on the availability of data. 
	\item \textbf{Computation:} Hardware advancement etc. The process of deep learning is extremely iterative, so we need to do the computations fast enough in order to work effectively. 
	\item \textbf{Algorithms:} Algorithmic innovations. Example: switching from sigmoid, which slows down the learning process due to it's gradient's being small, to ReLU. Switching to ReLU made the gradient descent algorithm much more efficient. 
\end{itemize}
